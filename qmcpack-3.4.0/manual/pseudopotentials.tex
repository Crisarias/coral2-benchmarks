\section{Obtaining pseudopotentials}

\subsection{Opium}
\label{subsec:opium}

Opium is a pseudopotential generation code available from the website \url{http://opium.sourceforge.net/}.  Opium can generate pseudopotentials with either Hartree-Fock or DFT methods.  Once you have a useable pseudopotential param file (for example, \texttt{Li.param}), generate pseudopotentials for use in Quantum ESPRESSO with the \texttt{upf} format as follows:
\begin{lstlisting}[caption=Generate UPF-formatted pseudopotential with Opium]
opium Li.param Li.log all upf
\end{lstlisting}
This generates a UPF-formatted pseudopotential (\texttt{Li.upf}, in this case) for use in Quantum ESPRESSO.  The pseudopotential conversion tool \texttt{ppconvert} can then convert UPF to FSAtom xml format for use in QMCPACK:
\begin{lstlisting}[caption=Convert UPF-formatted pseudopotential to FSAtom xml format]
ppconvert --upf_pot Li.upf --xml Li.xml
\end{lstlisting}

\subsection{Burkatzki-Filippi-Dolg}
\label{subsec:BFD}

Burkatzki \textit{et al.} developed a set of energy-consistent pseudopotenitals for use in QMC~\cite{Burkatzki07,Burkatzki08}, available at \url{http://www.burkatzki.com/pseudos/index.2.html}.  To convert for use in QMCPACK, select a pseudopotential (choice of basis set is irrelevant to conversion) in GAMESS format and copy the ending (pseudopotential) lines beginning with(element symbol)-QMC GEN:

\begin{lstlisting}[caption=BFD Li pseudopotential in GAMESS format]
Li-QMC GEN 2 1
3
1.00000000 1 5.41040609
5.41040609 3 2.70520138
-4.60151975 2 2.07005488
1
7.09172172 2 1.34319829
\end{lstlisting}
Save these lines to a file (here, named \texttt{Li.BFD.gamess}; the exact name may be anything as long as it is passed to \texttt{ppconvert} after --gamess\_pot).  Then, convert using \texttt{ppconvert} with the following:
\begin{lstlisting}[caption=Convert GAMESS-formatted pseudopotential to FSAtom xml format]
  ppconvert --gamess_pot Li.BFD.gamess --s_ref "2s(1)" --p_ref "2p(0)" --xml Li.BFD.xml
\end{lstlisting}
\begin{lstlisting}[caption=Convert GAMESS-formatted pseudopotential to Quantum ESPRESSO UPF format]
  ppconvert --gamess_pot Li.BFD.gamess --s_ref "2s(1)" --p_ref "2p(0)" --log_grid --upf Li.BFD.upf
\end{lstlisting}

\subsection{CASINO}
\label{subsec:CASINO}
The QMC code CASINO also makes available its pseudopotentials available at the website \url{https://vallico.net/casinoqmc/pplib/}. To use one in QMCPACK, select a pseudopotential and download its summary file (\texttt{summary.txt}), its tabulated form (\texttt{pp.data}), and (for ppconvert to construct the projectors to convert to Quantum ESPRESSO's UPF format) a CASINO atomic wavefunction for each angular momentum channel (\texttt{awfn.data\_*}).  Then, to convert using ppconvert, issue the following command:
\begin{lstlisting}[caption=Convert CASINO-formatted pseudopotential to Quantum ESPRESSO UPF format]
ppconvert --casino_pot pp.data --casino_us awfn.data_s1_2S --casino_up awfn.data_p1_2P --casino_ud awfn.data_d1_2D --upf Li.TN-DF.upf
\end{lstlisting}
QMCPACK can directly read in the CASINO-formated pseudopotential (\texttt{pp.data}), but four parameters found in the pseudopotential summary file must be specified in the pseudo element (\texttt{l-local}, \texttt{lmax}, \texttt{nrule}, \texttt{cutoff})[see Section~\ref{subsec:pseudopotentials} for details]:
\begin{lstlisting}[caption=XML syntax to use CASINO-formatted pseudopotentials in QMCPACK]
<pairpot type="pseudo" name="PseudoPot" source="ion0" wavefunction="psi0" format="xml">
   <pseudo elementType="Li" href="Li.pp.data" format="casino" l-local="s" lmax="2" nrule="2" cutoff="2.19"/>
   <pseudo elementType="H" href="H.pp.data" format="casino" l-local="s" lmax="2" nrule="2" cutoff="0.5"/>
</pairpot>
\end{lstlisting}
